% Created 2018-06-21 Thu 12:30
\documentclass[8pt]{beamer}
\usepackage[sc,osf]{mathpazo}   % With old-style figures and real smallcaps.
\linespread{1.025}              % Palatino leads a little more leading
% Euler for math and numbers
\usepackage[euler-digits,small]{eulervm}
%\documentclass[10pt]{llncs}
%\usepackage{llncsdoc}
\usepackage{hyperref}
\usepackage{minted}
\usemintedstyle{xcode}
\usepackage[utf8]{inputenc}
\usepackage[T1]{fontenc}
\usepackage{fixltx2e}
\usepackage{graphicx}
\usepackage{longtable}
\usepackage{float}
\usepackage{wrapfig}
\usepackage{rotating}
\usepackage[normalem]{ulem}
\usepackage{amsmath}
\usepackage{textcomp}
\usepackage{marvosym}
\usepackage{wasysym}
\usepackage{amssymb}
\usepackage{polynom}
\usepackage{changepage}
\usepackage{lipsum}

\hypersetup{colorlinks=true,
    linkcolor = blue,
    urlcolor  = blue,
    citecolor = blue,
    anchorcolor = blue
}
\renewcommand{\mod}[1]{\left( \texttt{mod}~#1 \right)}
\newcommand{\cpp}[1]{\mintinline{cpp}{#1}}
\newcommand{\py}[1]{\mintinline{py}{#1}}
\newcommand{\raw}[1]{\mintinline{text}{#1}}
\newcommand{\hs}[1]{\mintinline{hs}{#1}}
\newcommand{\smallpt}{\texttt{smallpt}}
\tolerance=1000
\usetheme{Antibes}
\author{Daven Scies, Siddharth Bhat}
\date{November 4th, 2020}
\institute{Haskell Exchange}
\title{Optimizing \smallpt}
\hypersetup{
  pdfkeywords={},
  pdfsubject={},
  pdfcreator={Emacs 24.5.1 (Org mode 8.2.10)}}
\begin{document}

\maketitle

\begin{frame}[fragile]{What is smallpt anyway?}
% \begin{adjustwidth}{-5em}{-5em}
\begin{minted}{cpp}
#include <math.h>
#include <stdlib.h>
#include <stdio.h>
struct Vec {      
  double x, y, z; // position, also color (r,g,b) 
  ... methods...
}; 
struct Ray { Vec o, d; Ray(Vec o_, Vec d_) : o(o_), d(d_) {} }; 
enum Refl_t { DIFF, SPEC, REFR };  // material types, used in radiance() 
struct Sphere { 
  double rad;   // radius 
  Vec p, e, c;  // position, emission, color 
  Refl_t refl;  // reflection type (DIFFuse, SPECular, REFRactive) 
  ... methods ...
  double intersect(const Ray &r) const // returns distance, 0 if nohit 
}; 
Sphere spheres[] = {//Scene: radius, position, emission, color, material 
  Sphere(1e5, Vec( 1e5+1,40.8,81.6), Vec(),Vec(.75,.25,.25),DIFF),//Left 
  ... initialization ...
}; 
inline bool intersect(const Ray &r, double &t, int &id) 
\end{minted}
% \end{adjustwidth}
\end{frame}


\begin{frame}[fragile]{What is smallpt anyway?}
\footnotesize
\begin{minted}{cpp}
Vec radiance(const Ray &r, int depth, unsigned short *Xi){ 
  double t;                               // distance to intersection 
  int id=0;                               // id of intersected object 
  if (!intersect(r, t, id)) return Vec(); // if miss, return black 
  const Sphere &obj = spheres[id];        // the hit object 
  Vec x=r.o+r.d*t, n=(x-obj.p).norm(), nl=n.dot(r.d)<0?n:n*-1, f=obj.c; 
  double p = f.x>f.y && f.x>f.z ? f.x : f.y>f.z ? f.y : f.z; // max refl 
  if (++depth>5) if (erand48(Xi)<p) f=f*(1/p); else return obj.e; //R.R. 
  if (obj.refl == DIFF){                  // Ideal DIFFUSE reflection 
    double r1=2*M_PI*erand48(Xi), r2=erand48(Xi), r2s=sqrt(r2); 
    Vec w=nl, u=((fabs(w.x)>.1?Vec(0,1):Vec(1))%w).norm(), v=w%u; 
    Vec d = (u*cos(r1)*r2s + v*sin(r1)*r2s + w*sqrt(1-r2)).norm(); 
    return obj.e + f.mult(radiance(Ray(x,d),depth,Xi)); 
  } else if (obj.refl == SPEC)            // Ideal SPECULAR reflection 
    return obj.e + f.mult(radiance(Ray(x,r.d-n*2*n.dot(r.d)),depth,Xi)); 
  Ray reflRay(x, r.d-n*2*n.dot(r.d));     // Ideal dielectric REFRACTION 
  bool into = n.dot(nl)>0;                // Ray from outside going in? 
  double nc=1, nt=1.5, nnt=into?nc/nt:nt/nc, ddn=r.d.dot(nl), cos2t; 
  if ((cos2t=1-nnt*nnt*(1-ddn*ddn))<0)    // Total internal reflection 
    return obj.e + f.mult(radiance(reflRay,depth,Xi)); 
  Vec tdir = (r.d*nnt - n*((into?1:-1)*(ddn*nnt+sqrt(cos2t)))).norm(); 
  double a=nt-nc, b=nt+nc, R0=a*a/(b*b), c = 1-(into?-ddn:tdir.dot(n)); 
  double Re=R0+(1-R0)*c*c*c*c*c,Tr=1-Re,P=.25+.5*Re,RP=Re/P,TP=Tr/(1-P); 
  return obj.e + f.mult(depth>2 ? (erand48(Xi)<P ?   // Russian roulette 
    radiance(reflRay,depth,Xi)*RP:radiance(Ray(x,tdir),depth,Xi)*TP) : 
    radiance(reflRay,depth,Xi)*Re+radiance(Ray(x,tdir),depth,Xi)*Tr); 
} 
\end{minted}
\end{frame}

\begin{frame}[fragile]{Haskell: the first stab}
\end{frame}

\begin{frame}[fragile]{Optimisation 1: }
\end{frame}

\begin{frame}[fragile]{Optimisation 2: }
\end{frame}

\begin{frame}[fragile]{Optimisation 3: }
\end{frame}
\begin{frame}[fragile]{Takeaways}
\pause
\begin{itemize}
\item Haskell can be fast \pause
\item ... with a lot of work! \pause
\item Accumulate optimizations to accrue performance wins.
\end{itemize}
\end{frame}
\end{document}
